% Размер страницы и шрифта
\documentclass[12pt,a4paper]{article}

% Работа с русским языком

%\usepackage[T2A]{fontenc}           % кодировка
\usepackage{cmap}                     % поиск в PDF
\usepackage[T2A]{fontenc}             % кодировка
\usepackage[utf8]{inputenc}           % кодировка исходного текста
\usepackage[russian, english]{babel}  % локализация и переносы


% Размер полей
\usepackage[top=0.5in, bottom=0.75in, left=0.625in, right=0.625in]{geometry}

%\usepackage{titlesec}               % Изменение формата заголовков

\begin{document}

Сентенция - судебный приговор военного суда или по уголовному делу. Краткое, остроумное изреченье (Даль).

Сонет - лирическое стихотворение, состоящее мз четырнадцати строк: двух четверостиший и двух трёхстиший.

Роман - большое повествовательное художественное произведение со сложным сюжетом, с большим числом действующих лиц, обычно в прозе.
Энциклопедический комментарий:
Роман - эпическое произведение, в котором повествование сосредоточнго на судьбе отдельной личности и ее отношении к окружающему мру, на становлении и развитии ее характера. Роман представляет индивидуальную и общественную жизнь как относитеьно самостоятельные, не исчерпывающие и не поглощающие друг друга стихии, и в этом состоит определюяющая особенность его жанорового содержания. Первой историческо формой европейского романа, зародившейся в  Испании, считают плутовской роман. В XVIII в. развиваются две основные разновидности: социально-бытовой (Г.Филдинг, Т.Смоллетт) и психологический роман (С.Ричардсон, Ж. Ж. Руссо, Л.Стерн, И.В.Гете). Романтики создают исторический роман (В. Скотт). В 1830-е гг. начинается классичексая эпоха социально-психологического романа критического реализма (Стендаль, О.Бальзак, Г.Флобер, Л. Н. Толстой, Ф. М. Достоевский и др.).

Литота - преуменьшение.

Куртуазный - изысканно вежливый; любезный, учтивый, галантный (Ефремова).

Каверза - (1) подвох, интрига; происки. (2) скрытая трудность, грозящая осложнениями, неприятностями (Ефремова).

Энигма - нечто непонятное; загадка (Ефоремова).

\end{document}