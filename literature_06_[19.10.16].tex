% Размер страницы и шрифта
\documentclass[12pt,a4paper]{article}

% Работа с русским языком

%\usepackage[T2A]{fontenc}           % кодировка
\usepackage{cmap}                     % поиск в PDF
\usepackage[T2A]{fontenc}             % кодировка
\usepackage[utf8]{inputenc}           % кодировка исходного текста
\usepackage[russian, english]{babel}  % локализация и переносы


% Размер полей
\usepackage[top=0.5in, bottom=0.75in, left=0.625in, right=0.625in]{geometry}

%\usepackage{titlesec}               % Изменение формата заголовков

\begin{document}

\subsection{Первая:}

...

\subsection{Вторая:}

Граф Нотр-Дамский, Сервантес.
По аналогии с работой Ницше "Рождение трагедии..." озаглавим наш... "Рождение романа из духа смеха".
До Дон Кихота роман являлся сказочным произведением. В современном понимании роман - это... (см. определение в словаре).
Здесь происходит цитирование Дон Кихота.
"Ни черта не стоят!
Смею вас заверить, ваше высокородие...
Жизнь Хинеса..."
Роман имеет дело с незавршенной современностью. В Дон Кихоте смех используется для... Многие персонажи брались на основе древних мифов (Гамлет, король Лир). Дон Кихот является абсолютно новым героем, до него ничего подобного не было. Дон Кихот пришёл, чтобы воплотить вымысел в правду.
Во втором томе, 26 гл. Дон Кихот в трактире. Там он видит спектакль - романс. У главного героя происходит диссонанс. Важным является то, что сначала он направляет рассказчика, а в конце он с мечом набрасывается на кукол, кроша их. Дон Кихот противопоставляет комизм действительности.
Комическому анализу подвергается всё.
Второе начало возрождения - начало героических и сверхгероических страстей.
Границы личности не очерчены. Достоевский: "Всё дозволенное есть шабаш".
Жордано Бруно.
Два Чуда: 
Испанский театр, который за 50 лет невероятно вырос (великие елизаветинские поэты).

\subsection{Семинар:}

Он и статью отличается, и величие абсолютный синкретизм, единство личности. Гамлет находится в другой ситуации, по сравнению с отцом. В отношениях Гамлета и Фортинбраса (перед отъездом, Фортинбрас идёт в поход, всё время в походе - динамическая фигура). "Земли, которую я завоюю, даже не хватит, чтобы похоронить воинов". Гамлет, корит себя. Фортинбрас принадлежит к рыцарскому времени - он воюет за клочок сена, для него кодекс важнее сути, но он рассуждает. У него соервенно новое сознание - он не верит на слово, ему приходится проверять информацию. Столкновение Лаэрта и Гамлета происходит на похоронах. Фортинбрас - литота. Лаэрт - гипербола. Лаэрт устраивает ритуал, это бесит Гамлета. Гамлет завидует Лаэрту, у которого не расщеплённое сознание. Гамлетовский мир не делится на белое и чёрное, у него масса оговорок.
Гамлет, как деятель:
анализ - нечто неслыханное и новое. Изначально он не верит ничему. Благодаря маске сумасшествия Гамлет получает сразу несколько дивидентов. Он переигрывает актерством короля. Он хочет устроить суд над королём.

\subsection{ДЗ:}

Жизнь есть сон - ...

"Страдания юного Вертера" - Гёте

\subsection{Словарь:}

Литота, роман, куртуазный, каверза, энигма.

\end{document}