% Размер страницы и шрифта
\documentclass[12pt,a4paper]{article}

% Работа с русским языком

%\usepackage[T2A]{fontenc}           % кодировка
\usepackage{cmap}                     % поиск в PDF
\usepackage[T2A]{fontenc}             % кодировка
\usepackage[utf8]{inputenc}           % кодировка исходного текста
\usepackage[russian, english]{babel}  % локализация и переносы


% Размер полей
\usepackage[top=0.5in, bottom=0.75in, left=0.625in, right=0.625in]{geometry}

%\usepackage{titlesec}               % Изменение формата заголовков

\begin{document}

\subsection{Первая:}

Испанская народная поговрка - "Собака на сене" - так называют пьесу. Все герои являются поэтами, они говорят афоризмами. Диана мучается и мучает Теодора своей любовью. Она сначала даёт ему надежду, потом отталкивает. Фабио - это низшая ступенька. Икару уподабливаются все герои - все они пытаются взлететь, рискуя опалить крылья.

...Два четверостишия на одну пару рифм и два..
Сонет - сложная форма. требующая изощренности, на любого поэта рубежа 16-17 века давит. Возникает страх влияния перед выдающимися предшественниками, например, Петраткрой. Его приемы так и просятся на подражание.
Английские сонетисты рубежа 16-17 века приняли вызов сонета. Драма отношений, сближения - отдаления - одни из немногих приемов сонета. Испоьзование пародий, метафор з приемы, используемые сэром ... Что значит слово в сонете? Здесь каждый элемент особенно важен:

1I never drank of Aganippe well...

I never drank of Aganippe well,
Nor ever did in shade of Tempe sit,
And Muses scorn with vulgar brains to dwell:
Poor layman I, for sacred rites unfit.
Some do I hear of poets' Fury tell,
But (God wot) wot not what they mean by it;
And this I swear, by blackest brook of hell,
I am no pick-purse of another's wit.
How falls it then, that with so smooth an ease
My thoughts I speak, and what I speak doth flow
In verse, and that my verse best wits doth
please?
Guess we the cause: 'What, is it thus?' Fie, no;
'Or so?' Much less. 'How then?' Sure, thus it is:
My lips are sweet, inspired with Stella's kiss.

Во втором керсете он меняет интонацию.
В английской литературе поэты научились виртуозно писат сонеты.

\subsection{Вторая:}

Лед и пламя - традиционная антитеза. Спенсер 

(Зачитывает сонет)

My Love is like to ice, and I to fire: 
How comes it then that this her cold so great 
Is not dissolved through my so hot desire, 
But harder grows the more I her entreat? 
Or how comes it that my exceeding heat 
Is not allayed by her heart-frozen cold, 
But that I burn much more in boiling sweat, 
And feel my flames augmented manifold? 
What more miraculous thing may be told, 
That fire, which all things melts, should harden ice, 
And ice, which is congeal’d with senseless cold, 
Should kindle fire by wonderful device? 
Such is the power of love in gentle mind, 
That it can alter all the course of kind.

Edmund Spenser

Начинается традиционно, строчка делится на две половины, в одну сторону лёд, в другую - огонь. Далее: "Как так получается, что этот ее крепкий холод". Чем больше люблю её, тем холоднее становтся она. Второй катрен симметричен первому, но он сиьнее первого, словечные средства сильнее. В первом "more", во втором - "much more". Английский сонет состоит из трех катренов и одного сонетного ключа. 8 и 6 - разделение на две части. В третьем сонете объединяются темы первого и второго. Испольуя повторение, автор выводит антитезу на другой уровень. Какова сила любви в благородной душе. Вместо того, чтобы сетовать на недоступность возлюбленной, он восхищается силой любви, которая мняет законы природы. На протяжении всего произведения силы льда и огня смешиваются в танце и достигают высот платонизма. Шекспир завершает "великую" эпоху английского сонета, для него Петрарка не указ. Наиболее удачными считаются переводы Маршака, которые, на самом деле, не такие уж и удачные. 

Дальше лектор усыпил меня своим нудным рассказом..

"Гремит лишь то, что пусто изнутри" - Шекспир.

\subsection{ДЗ:}



\subsection{Словарь:}

Сентенция, сонет.

\end{document}