% Размер страницы и шрифта
\documentclass[12pt,a4paper]{article}

% Работа с русским языком

%\usepackage[T2A]{fontenc}           % кодировка
\usepackage{cmap}                     % поиск в PDF
\usepackage[T2A]{fontenc}             % кодировка
\usepackage[utf8]{inputenc}           % кодировка исходного текста
\usepackage[russian, english]{babel}  % локализация и переносы


% Размер полей
\usepackage[top=0.5in, bottom=0.75in, left=0.625in, right=0.625in]{geometry}

%\usepackage{titlesec}               % Изменение формата заголовков

\begin{document}

{\huge История литературы}
\section{Эпоха возрождения}
Ранее возрождение и предвозрождение. 4 важных места:
Базилика Сан Франциска в городе (?). ответветственно,   важная фигура -- Святой Франциск.
Святой Франциск
Отказался от всего имущества, и назвал себя по латыни   иота эт литератус, "низший из всех
низших". И пошел скитаться по миру. Почему он так ут?   Впервые за долгое время он смог
соответствовать слову делом, что было редкостью в то   емя (где-то 12 век), во время
религиозного кризиса. Жизнью своей воплотил идею жизни    Христе (не иметь имущества,
переходит от места к месту, питаться тем, что бог   шлет). Поэтому это можно считать
началом нового времени: человек отказывается жить в   аром мире и начинает жить в
новом. При этом он был всеми любим, никем не гоним и   кем не ненавидем. Он
признавался и в народе, и в официально, Церковью. этому   его до сих пор весьма уважают
и почитают за чудо. Святой Франциск возродил ангелие,   живое и дышащее, здесь и сейчас.
Virtu -- доблесть, одновременно талант, одновременно   пульс.
Тот же Диоген дил в бочке напоказ, в отличие от   анциска, однако поступок ничем не
менее оригинальный. Проявление индивидуальности.   ассический пример -- обращение к
каждому предмету и животному, обращение "брат" к лку,   проповедь птичкам. И, конечно
же, к людям -- жгучая любовь к близким.
Теперь обратно к базилике. Кто ее построил? Ну там ло   чье-то имя, но я не разобрала. Был
учеником (другом) Франциска. Невероятно образованный,   пломат, друг еще и императора,
инженер, блаблабла... В общем, крут. И, собственно,   строил храм -- первое следствие
похождений Франциска.
Еще: какая-то фреска Чемогуева (?), где тоже Франциск.    иконизированное, не
облагороженное, скорее юридивое, и очень живое,   дивидуальное.
Фрески все в том же храме. У них появляется объем --   ктически начало новой школы, хотя
еще не совсем. Объем -- заявление о себе, о своем   исутствии, о своей индивидуальности.
Итого, эта базилика -- первый шаг к Возрождению.
В доказательство ключевого значения Франциска: три   новные личности Возрождения --
Данте, Рабле и Сервантас -- так или иначе связаны (ишли   из) францисканского ордена. По
крайней мере, Данте и Сервантас были нахами-мирянами,   или как-то так.
Homo universalis -- человек, охватывающий все, что жно,   знающий все. Ренессанский
вариант человека.
Францисканцы-номиналисты уделяли особое внимание   дивидуальности каждой вещи.
Еще там был какй-то чувак Вильгельм Баскервильский,   торому помогал монах Атцон
(отсылка к Шерлоку, в общем). Расследовал, и еще тел   найти ту самую вторую книгу
Аристотеля, о комедии, чтобы пусть смех в мир. И ему   отивостоял монах... А может, это все
из Умберто Эко, я что-то запуталась.
Второе место:
Данте. Сан Миньято во Флоренции. Везде цитатки из медии.
Данте не начало Ренессанса, но точно в преддверии.
Что такое коммуна (кажется, речь только о Флоренции, и   Италии...) в то время? Ну
во-первых, это довольно сильная организация. И ловия   процветающих городов прекрасно
на них влияют. Но вообще, я путаюсь, сложно за ваком   записывать =( Флоренция, Двельфы,
Гибелины... Что-то там сражаются... Одни за папу, угие   за императора... Данте изгоняют из
Флоренции... В общем, страшное время, но в то же время    время гордости и
невообразимого самовыражения, люди разучаются аняться.   Гордыня, возвеличивания
самого себя, прочее -- все это становится все более ко   выражено к 14-15 веку.
Какая-то фреска, какой-то чувак Сигизмунд Дурная лова,   коленопреклонный перед святым
Сигизмундом -- но при этом с гордой позой и любимой   бакой.
Эта гордость -- первое. Второе -- комунны всегда   аждуют. При этом, по скольку надо быть
лучше во всем, они соревновались в том числе и в   кусстве, так что даже самые
кровожадные заманивали к себе поэтов, художников, еных   и прочее... Там приводили в
пример какого-то конкретного чувака, но я имя не овила.   Кажется, тот же Сигизмунд.
Время городов-государств. Сильнейшее -- Флоренция.   ожество людей искусства пришли
именно оттуда.
Кстати, в Комедии Данте множество существ терсуются,   как там дела во Флоренции. Что
тоже многое говорит о ее роли.
\section{Мой конспект}
\subsection{Первая пара}
Появление университетов (от лат. universitas - совокупность). Считается, что Болонский университет (основан в 1088 г.) был самым первым. Первоначально университеты часто возникали на основе церковных школ. Они готовили специалистов по философии, богословию, праву, медицине. Также там изучали научные труды. У университета должна быть своя грамота. В основе университета лежала равноправная система отношений. \\
"Historia Calamitatum Mearum" - Пьер Абеляр (история моих бедствий - автобиографическое произведение).\\
Пьер Абеляр - средневековый французский философ-схоласт, теолог, поэт.\\
Схоластика - средневековая философия, характеризующаяся соединением теолого-догматических предпосылок с рационалистической методикой и интересом к формально-логическим проблемам. Формальное знание, оторванное от жизни и практики.\\
Основные направления изучения философии в университете: номинализм и реализм - противоборствующие направления (я что-то не догнал, чем они различаются, в другой раз посмотрю!). \\
"Имя Розы" - Umberto Eco. Первая чась поэтики Аристотеля (вторая не сохранилась). Поэзия вагантов (от лат. vagantes - странствующие) анонимна. Ваганты - бродячие поэты, использующие в своих произведениях преимущественно латинский язык - международный сословный язык духовенства.
\subsection{Вторая пара}
Ренессанс, возрождение (фр. Renaissance, итал. Rinascimento; от re/ri — «снова» или «заново» + nasci — «рождённый») - эпоха, пришедшая в Европе на смену Средним векам. Расцветает интерес к античной культуре, происходит как бы её «возрождение» — так и появился термин.\\
Базилика (тип строения прямоугольной формы, которое состоит из нечётного числа (1, 3 или 5) различных по высоте нефов) Св. Франциска находится в Ассизи, там же его могила. Он провозгласил себя низшим из низших и пошёл скитаться по миру. Франциск был не таким, как все - его слова соответствовали делу. Virtu - доблесть. Сан Миньято.
\subsection{Семинар}
(Обсуждаем пятую песнь Божественной комедии Данте). Для понимания замысла ответим на три вопроса: за что наказаны любящие, в чём причина падения любящих, отношение Данте к Франческе. Нам предстают три женщины: Дидона, Семирамида, Клеопатра. Все они наказаны за сладострастие. Дидона возглавляет эту троицу, т.к. её грех считается самым тяжелым - она пыталась помешать Энею создать Рим (ключевой город для Данте). Три мужчины перечисляются: Ахилл, Парис, Тристан. Затем Паула и Франческа отделяются от вихря, символизирующего вихрь в голове влюбленных (метафора вихря), и приближаются к Данте. У Данте эмпатия к этим героям (эмпатия - способность входить в чужое эмоциональное состояние, сопереживать).

\end{document}